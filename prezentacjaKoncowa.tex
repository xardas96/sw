\documentclass{beamer}

\mode<presentation>
{
\usetheme{Warsaw}
\setbeamercovered{transparent}
}

\usepackage[english, polish]{babel}
\usepackage[cp1250]{inputenc}
\usepackage{polski}

\usepackage{mathptmx}
\usepackage[scaled=.90]{helvet}
\usepackage{courier}

\usepackage[T1]{fontenc}
\usepackage{graphicx}
\usepackage{float}

\title{Systemy wizyjne \\ \textbf{Wspomagana rzeczywisto��}}
\subtitle{Prezentacja ko�cowa}

\author{Jacek Miiller, Roman Sorokowski}

\institute[Universities of]
{Informatyka \\ Wydzia� Informatyki i Zarz�dzania \\ Politechnika
Wroc�awska}

\date{13 stycznia 2014}

\begin{document}

\begin{frame}
\titlepage
\end{frame}

\section{Wst�p}
\begin{frame}
\frametitle{Plan prezentacji}
\begin{itemize}
  \item Wizja systemu
  \item Etapy przetwa�ania obrazu
  \begin{itemize}
    \item Wykrywanie markera
    \item Ustalenie orientacji
    \item Dekodowanie markera
    \item Wy�wietlanie modelu
  \end{itemize}
  \item Plany dalszego rozwoju
  \item Prezentacja systemu
\end{itemize}
\end{frame}

\section{Wizja systemu}
\begin{frame}
\frametitle{Wizja systemu}
\begin{itemize}
  \item Dzia�anie w czasie rzeczywistym
  \item Wykrywanie marker�w, przy r�nych warunkach o�wietlenia i nachylenia
  \item Wy�wietlanie obiekt�w 3D w oparciu o kod zawarty w markerze
\end{itemize}
\end{frame}

\section{Etapy przetwa�ania obrazu}
\subsection{Wykrywanie markera}
\begin{frame}
\frametitle{Obraz z kamery}
\begin{figure}
	\centering
	\includegraphics[scale=0.5]{stage0}
\end{figure}
\end{frame}

\begin{frame}
\frametitle{Etap 1a - Konwolucja pochodn� Gaussa}
\begin{columns}[onlytextwidth]
\begin{column}{5cm}
\centering
	\includegraphics[scale=0.3]{stage0x}
\end{column}
\begin{column}{5cm}
\begin{figure}
	\centering
	\includegraphics[scale=0.3]{stage0y}
\end{figure}
\end{column}
	
\end{columns}
\end{frame}

\begin{frame}
\frametitle{Etap 1b - Punkty kraw�dziowe}
\begin{figure}
	\centering
	\includegraphics[scale=0.5]{stage1}
\end{figure}
\end{frame}

\begin{frame}
\frametitle{Etap 2 - ��czenie punkt�w w regionach}
\begin{figure}
	\centering
	\includegraphics[scale=0.5]{stage2}
\end{figure}
\end{frame}

\begin{frame}
\frametitle{Etap 3 - ��czenie linii}
\begin{figure}
	\centering
	\includegraphics[scale=0.5]{stage3}
\end{figure}
\end{frame}

\begin{frame}
\frametitle{Etap 4 - Wyd�u�anie linii}
\begin{figure}
	\centering
	\includegraphics[scale=0.5]{stage4}
\end{figure}
\end{frame}

\begin{frame}
\frametitle{Etap 5 - Filtrowanie linii bez k�t�w}
\begin{figure}
	\centering
	\includegraphics[scale=0.5]{stage5}
\end{figure}
\end{frame}

\begin{frame}
\frametitle{Etap 6 - Wykrycie marker�w}
\begin{figure}
	\centering
	\includegraphics[scale=0.5]{stage6}
\end{figure}
\end{frame}

\begin{frame}
\frametitle{Etap 7 - Filtrowanie marker�w}
\begin{figure}
	\centering
	\includegraphics[scale=0.5]{stage7}
\end{figure}
\end{frame}

\subsection{Ustalenie orientacji}
\begin{frame}
\frametitle{Ustalanie orientacji}
\begin{figure}
	\centering
	\includegraphics[scale=0.5]{stage8}
\end{figure}
\end{frame}

\subsection{Dekodowanie markera}
\begin{frame}
\frametitle{Dekodowanie markera}
\begin{figure}
	\centering
	\includegraphics[scale=1.5]{stage8-5}
\end{figure}
\end{frame}

\subsection{Wy�wietlanie modelu}

\begin{frame}
\frametitle{Ustalanie pozycji kamery}
\begin{itemize}
  \item wyznaczenie projekcji punktu P w przestrze� 3D
	\begin{equation}
		p = \left[R|t\right]P 
	\end{equation}
  \item rzutowanie punktu P w przestrze� 2D
  	\begin{equation}
		q = Kp
	\end{equation}
  \item macierz kalibracji kamery
 	\begin{equation}
		K = 
		\left[
		\begin{array}{ccc}
			f & 0 & c_x \\
			0 & f & c_y \\
			0 & 0 & 1 \\
		\end{array}
		\right]
	\end{equation}
   \item algorytm PnP
\end{itemize}
\end{frame}

\begin{frame}
\frametitle{Wy�wietlanie modelu}
\begin{figure}
	\centering
	\includegraphics[scale=1]{stage9b}
\end{figure}
\end{frame}

\section{Plany dalszego rozwoju}
\begin{frame}
\frametitle{Plany dalszego rozwoju}
\begin{itemize}
  \item optymalizacja - przyspieszenie dzia�ania, wielow�tkowo��
  \item odporno�� na szumy - warunki o�wietleniowe, niedoskona�o�ci kamery
  \item platformy mobilne
\end{itemize}
\end{frame}

\section{Zako�czenie}
\begin{frame}
\frametitle{Zako�czenie}
	\centering
	\textbf{Dzi�kujemy za uwag�!}
\end{frame}

\end{document}
